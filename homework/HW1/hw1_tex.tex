\documentclass[12pt]{article}
% TikZ setup
\usepackage{tikz}
\usetikzlibrary{arrows.meta,positioning,bending}
\newcommand{\duedate}{10/06/2025 2pm}
\newcommand{\assignment}{Assignment 1} % Change to "Problem Set X"

% Change the following to your name and UNI.
\newcommand{\name}{Aksel Kretsinger-Walters, adk2164}
\newcommand{\email}{adk2164@columbia.edu}

% NOTE: Defining collaborators is optional; to not list collaborators, comment out the
% line below. Maximum of two collaborators per problem set.
%\newcommand{\collaborators}{Vig Vigerton (\texttt{UNI}), Alice Bob (\texttt{UNI})}
% No collaborators on PS 0

\makeatletter
\def\input@path{{../}{../../}} % add as many parents as you need
\makeatother
% Copyright 2021 Paolo Adajar (padajar.com, paoloadajar@mit.edu)
%
% Permission is hereby granted, free of charge, to any person obtaining a copy of this
% software and associated documentation files (the "Software"), to deal in the Software
% without restriction, including without limitation the rights to use, copy, modify,
% merge, publish, distribute, sublicense, and/or sell copies of the Software, and to
% permit persons to whom the Software is furnished to do so, subject to the following conditions:
%
% The above copyright notice and this permission notice shall be included in all copies
% or substantial portions of the Software.
%
% THE SOFTWARE IS PROVIDED "AS IS", WITHOUT WARRANTY OF ANY KIND, EXPRESS OR IMPLIED,
% INCLUDING BUT NOT LIMITED TO THE WARRANTIES OF MERCHANTABILITY, FITNESS FOR A
% PARTICULAR PURPOSE AND NONINFRINGEMENT. IN NO EVENT SHALL THE AUTHORS OR COPYRIGHT
% HOLDERS BE LIABLE FOR ANY CLAIM, DAMAGES OR OTHER LIABILITY, WHETHER IN AN ACTION OF
% CONTRACT, TORT OR OTHERWISE, ARISING FROM, OUT OF OR IN CONNECTION WITH THE SOFTWARE OR
% THE USE OR OTHER DEALINGS IN THE SOFTWARE.

%%%%%%%%%%%%%%%%%%%%%%%%%%%%%%%%%%%%%%
%%%%% DO NOT MODIFY THIS FILE %%%%%%%%
%%%%%%%%%%%%%%%%%%%%%%%%%%%%%%%%%%%%%%

%%%%%%%%%%%%%%%%%%%%%%%%%%%%%%
%%%%% CLASS SPECIFICS %%%%%%%%
%%%%%%%%%%%%%%%%%%%%%%%%%%%%%%
\newcommand{\classnum}{COMS 4776}
\newcommand{\subject}{Neural Nets and Deep Learning}
\newcommand{\instructors}{Richard Zemel}
\newcommand{\semester}{Fall 2025}

%%%%%%%%%%%%%%%%%%%%%%%%%%%%%%
%%%%% PACKAGE IMPORTS %%%%%%%%
%%%%%%%%%%%%%%%%%%%%%%%%%%%%%%
\usepackage{fullpage}
\usepackage{enumitem}
\usepackage{amsfonts, amssymb, amsmath,amsthm}
\usepackage{tikz}
\usepackage{hyperref}
\usepackage{ifthen}

\hypersetup{
		colorlinks=true,
		linkcolor=blue,
		filecolor=magenta,
		urlcolor=blue,
}

%%%%%%%%%%%%%%%%%%%%%%%%%%%%
%%%%% CUSTOM MACROS %%%%%%%%
%%%%%%%%%%%%%%%%%%%%%%%%%%%%
\usepackage{macros}

%%%%%%%%%%%%%%%%%%%%%%%%%
%%%%% FORMATTING %%%%%%%%
%%%%%%%%%%%%%%%%%%%%%%%%%
\setlength{\parindent}{0mm}
\setlength{\parskip}{2mm}

\setlist[enumerate]{label=({\alph*})}
\setlist[enumerate, 2]{label=({\roman*})}

\allowdisplaybreaks[1]

%%%%%%%%%%%%%%%%%%%%%
%%%%% HEADER %%%%%%%%
%%%%%%%%%%%%%%%%%%%%%
\newcommand{\psetheader}{
		\ifthenelse{\isundefined{\collaborators}}{
				\begin{center}
						{\setlength{\parindent}{0cm} \setlength{\parskip}{0mm}

								{\textbf{\subject} \hfill \name}

								\textbf{\assignment} \hfill \href{mailto:\email}{\tt \email}

								\classnum:~\semester \hfill \textbf{Due:}~\duedate~11:59 PM ET

						\hrulefill}
				\end{center}
		}{
				\begin{center}
						{\setlength{\parindent}{0cm} \setlength{\parskip}{0mm}

								{\textbf{\subject} \hfill \name\footnote{Collaborator(s): \collaborators}}

								\textbf{\assignment} \hfill \href{mailto:\email}{\tt \email}

								\classnum:~\semester \hfill \textbf{Due:}~\duedate~11:59 PM ET

						\hrulefill}
				\end{center}
		}
}

%%%%%%%%%%%%%%%%%%%%%%%%%%%%%%%%%%%
%%%%% THEOREM ENVIRONMENTS %%%%%%%%
%%%%%%%%%%%%%%%%%%%%%%%%%%%%%%%%%%%
\newtheorem{theorem}{Theorem}
\newtheorem{lemma}[theorem]{Lemma}
\newtheorem{example}[theorem]{Example}
 %% DO NOT CHANGE THIS LINE

\begin{document}

\psetheader %% DO NOT CHANGE THIS LINE

\section*{(1) Written Problems}
\subsection*{(1.1) Hard-Coding Networks}
So, for this problem, we need to create a neural network that can identify a superlative superincreasing
sequence. Our input vector is $\RR^{1\times 4}$, our hidden layer is $\RR^{1\times 3}$, and our output
layer is $\{0, 1\} \in \RR^{1\times 1}$. It's recommended to use the hard threshold activation function and the indicator
activation function.

My approach to this problem is to use each hidden neuron to test each step of the superincreasing sequence.
So the first neuron will test if $x_2 > x_1$, the second neuron will test if $x_3 > x_1 + x_2$,
and the third neuron will test if $x_4 > x_1 + x_2 + x_3$. Because superincreasing sequences must be
\tbf{strictly increasing}, ties will be treated as failures. The output neuron will then use the indicator
function to check that all three hidden neurons "passed" their tests.

Our hidden layer weights and biases are:
$$
\bW^{(1)} =
\begin{pmatrix}
		1 & -1 & 0 & 0 \\
		1 & 1 & -1 & 0 \\
		1 & 1 & 1 & -1
\end{pmatrix}
\quad
\bb^{(1)} =
\begin{pmatrix}
		0 \\
		0 \\
		0
\end{pmatrix}
\bw^{(2)} =
\begin{pmatrix}
		1 \\
		1 \\
		1
\end{pmatrix}
\quad
b^{(2)} = 1
$$

Our first hidden neurons are all using the hard threshold activation function, and our output
neuron is using the indicator activation function. From row 1, a value of 0 is considered a
pass, and a value of 1 is considered a fail. From row 2, any value outside of $[-1,1]$ is
a failure. Hence, the weights of second row are all 1s, and the bias is 1.

\newpage
\subsection*{(1.2) Backpropagation}
\tbf{Computational Graph}

% \medskip
\begin{tikzpicture}[
				>=Stealth,
				node distance=16mm and 18mm,
				var/.style={circle,draw,minimum size=7mm,inner sep=0pt},
				every edge quotes/.style={auto,red}
		]

		% Nodes
		\node[var] (X) {$x$};

		\node[var, right=of X, yshift=12mm] (a1) {$a_1$};
		\node[var, right=of X, yshift=-12mm] (a2) {$a_2$};

		\node[var, right=8mm of a2] (u) {$u$};

		\node[var, right=22mm of a1] (c1) {$c_1$};
		\node[var, right=11mm of u] (c2) {$c_2$};

		\node[var, right=15mm of c2] (g) {$g$};
		\node[var, right=22mm of g] (y) {$y$};
		\node[var, right=11mm of y] (yp) {$y'$};
		\node[var, right=12mm of yp] (s) {$s$};
		\node[var, right=12mm of s] (J) {$\cJ$};

		% External input t into s
		\node[var, above left=14mm of s, draw=none] (t) {$t$};

		% Edges (black structure)
		\draw[->] (X) -- (a1);
		\draw[->] (X) -- (a2);

		\draw[->] (a1) -- (u);
		\draw[->] (a2) -- (u);

		\draw[->] (a1) -- (c1);
		\draw[->] (u) -- (c2);

		\draw[->] (c1) -- (g);
		\draw[->] (c2) -- (g);

		\draw[->] (g) -- (y) node[midway, above, red] {$\bW^{(3)}$};
		\draw[->] (y) -- (yp);
		\draw[->] (yp) -- (s);
		\draw[->] (s) -- (J);

		% Skip from x to y (curved)
		\draw[->] (X) to[bend right=50] node[pos=0.6, below, red] {$\bW^{(4)}$} (y);

		% t -> s
		\draw[->] (t) -- (s);

		% Red parameter/bias annotations
		\node[red, above =0mm of a1] {$\bW^{(1)}$};
		\node[red, below=0mm of a1] {$b^{(1)}$};

		\node[red, above=0mm of a2] {$\bW^{(2)}$};
		\node[red, below=0mm of a2] {$b^{(2)}$};

		\node[red, above=1mm of g] {$\bW^{(g)}$};
		\node[red, below=0mm of g] {$b^{(g)}$};

\end{tikzpicture}

\tbf{Backwards Pass}
\begin{align*}
		\overline{\cJ} &= 1 \\
		\overline{\bs} &= -\overline{\cJ} = - 1 \cdot 1 = -1 \\
		\overline{\by'} &= \overline{\bs} (\by-\bt) = \bt-\by \\
		\overline{\by} &= \overline{\by'} \cdot \text{softmax}'(\by)  \\
		\overline{\bg} &= \bW^{(3)T}\cdot \overline{\by} \\
		\overline{\bW^{(3)}} &= \overline{\by}\cdot \bg^T \\
		\overline{\bW^{(g)}}&= \overline{\bg}\cdot
		\begin{pmatrix}
				\bc_1 \\
				\bc_2
		\end{pmatrix}^T \\
		\overline{b^{(g)}} &= \overline{\bg} \\
		\overline{c_1} &= \bW^{(g)}_{c_1} \cdot \overline{\bg}  \\
		\overline{c_2} &= \bW^{(g)}_{c_2} \cdot \overline{\bg}  \\
		\overline{\bu} &= \overline{c_2} \cdot \text{ReLU}'(c_2) \\
		\overline{\bW^{(2)}} &= \overline{\bu} \cdot \bx^T \\
		\overline{b^{(2)}} &= \overline{\ba_2} \\
		\overline{\ba_2} &= \overline{\bu} \\
		\overline{\bW^{(1)}} &= \overline{\ba_1} \cdot \bx^T \\
		\overline{b^{(1)}} &= \overline{\ba_1} \\
		\overline{\ba_1} &= \overline{\bu} + \overline{c_1} \cdot \sigma'(a_1) \\
		\overline{\bx} &= \bW^{(1)T} \cdot \overline{\ba_1} + \bW^{(2)T} \cdot \overline{\ba_2} + \bW^{(4)T} \cdot \overline{\by}
\end{align*}
\newpage
\subsection*{(1.3) Automatic Differentiation}
\tbf{Compute Hessian}

Our loss function is $$L(\bx) = \frac{1}{2}\bx^T\bv\bv^T\bx$$

A quick simplification (note that $\bv^T\bx$ is a scalar, so it's equal to its own transpose):
\begin{align*}
		L(\bx) &= \frac{1}{2}\bx^T\bv\bv^T\bx \\
		&= \frac{1}{2}(\bv^T\bx)(\bx^T\bv) \\
		&= \frac{1}{2}(\bv^T\bx)^2
\end{align*}
We can see that this is a quadratic form ($\frac{1}{2}\bx^T\bv\bv^T\bx = \frac{1}{2}\bx^T\bA\bx$).

So, the gradient is:
\begin{align*}
		\nabla L(\bx) &= \frac{1}{2}(\bA+\bA^T)\bx \\
		&= \frac{1}{2}(\bv\bv^T + \bv^T\bv)\bx \\
		&= \bv\bv^T\bx
\end{align*}

And the Hessian is:
\begin{align*}
		\nabla^2 L(\bx) &= \bA \\
		&= \bv\bv^T
\end{align*}

Now that we have the correct Hessian formula, lets plug in our values for $\bv$ and then solve for $\bx$
\begin{align*}
		\bv^T = [3,1,4] &\implies \bv =
		\begin{pmatrix}
				3 \\
				1 \\
				4
		\end{pmatrix} \\
		\nabla^2 L(\bx) &= \bv\bv^T \\
		&=
		\begin{pmatrix}
				3 \\
				1 \\
				4
		\end{pmatrix}
		\begin{pmatrix}
				3 & 1 & 4
		\end{pmatrix} \\
		&=
		\begin{pmatrix}
				9 & 3 & 12 \\	
				3 & 1 & 4 \\
				12 & 4 & 16
		\end{pmatrix}\\
		\bx^T = [2,1,3] &\implies \bx =
		\begin{pmatrix}
				2 \\
				1 \\
				3
		\end{pmatrix} \\
		\nabla^2 L(\bx) &=
		\begin{pmatrix}
				9 & 3 & 12 \\	
				3 & 1 & 4 \\
				12 & 4 & 16
		\end{pmatrix}
		\begin{pmatrix}
				2 \\
				1 \\
				3
		\end{pmatrix} \\
		&=
		\begin{pmatrix}
				9\cdot 2 + 3\cdot 1 + 12	\cdot 3 \\
				3\cdot 2 + 1\cdot 1 + 4 \cdot 3 \\
				12\cdot 2 + 4\cdot 1 + 16 \cdot 3
		\end{pmatrix} \\
		&=
		\begin{pmatrix}
				57 \\
				19 \\
				76
		\end{pmatrix}
\end{align*}
\tbf{Computation Cost}
The cost of explicitly computing the Hessian using automatic differentiation is $O(n^2)$, where $n$ is the
dimension of $\bx$. This is because the Hessian is an $n \times n$ matrix, and each entry
requires $O(1)$ time to compute. Therefore, the total time complexity is $O(n^2)$.

However, we could compute the Hessian-vector product
efficiently using the formula $\nabla^2 L(\bx) = \bv(\bv^T\bx)$. This would only require
$O(n)$ time, since we only need to compute the dot product $\bv^T\bx$ and then scale $\bv$ by that
scalar.

\subsection*{1.3.1 Vector-Hessian Products}
Hah, looks like I got a bit ahead of myself in the previous section. We'll explicitly compute the
vector-Hessian product here.
\begin{align*}
		\bz = \bH \by &= (\bv\bv^T)\by \\
		\bv^T &= [4,2,1] \\
		\by^T &= [2,2,1] \\
		\text{Reverse-mode autodiff:} \\
		\bv^T\by &= 4\cdot 2 + 2\cdot 2 + 1\cdot 1 = 13 \\
		\bz &= 13 \cdot
		\begin{pmatrix}
				4 \\
				2 \\
				1
		\end{pmatrix} =
		\begin{pmatrix}
				52 \\
				26 \\
				13
		\end{pmatrix} \\
		\text{Forward-mode autodiff:} \\
		\bv\bv^T &=
		\begin{pmatrix}
				16 & 8 & 4 \\
				8 & 4 & 2 \\
				4 & 2 & 1
		\end{pmatrix} \\
		\bz &=
		\begin{pmatrix}
				16 & 8 & 4 \\
				8 & 4 & 2 \\
				4 & 2 & 1
		\end{pmatrix}
		\begin{pmatrix}
				2 \\
				2 \\
				1
		\end{pmatrix} =
		\begin{pmatrix}
				52 \\
				26 \\
				13
		\end{pmatrix}
\end{align*}

\tbf{Computation Efficiency}
\begin{align*}
		\bH &= \bv \bv^T \\
		\bZ = \bH \by_1 \by_2^T &= \bv (\bv^T \by_1) \by_2^T
\end{align*}
Let's evaluate how many operations this takes using forward-mode and reverse-mode autodiff.
\begin{align*}
		\bv \in {\RR^n}, \by_1 &\in {\RR^n}, \by_2 \in {\RR^m} \\
		\tbf{Reverse-mode autodiff:} \\
		\by_2 \by_1^T =\bY \in \RR^{n \times m} &= O(n m) \text{ operations to compute} \\
		\bv^T \bY = \ba \in \RR^{1 \times m} &= O(n^2) \text{ operations to compute} \\
		\bv \ba = \bZ \in \RR^{n \times m} &= O(n m) \text{ operations to compute} \\
		\text{Total: } O(n m) + O(n^2) + O(n m) &= O(n^2 + n m) \text{ operations} \\
		&= O(n^2) \text{ if } m \leq n. \quad O(n m) \text{ if } m > n \\
		\tbf{Forward-mode} \\
		\bv \bv^T =\bH \in \RR^{n \times n} &= O(n^2) \text{ operations to compute} \\
		\bH \by_1 = \bz \in \RR^{n \times 1} &= O(n^2) \text{ operations to compute} \\
		\bz \by_2^T = \bZ \in \RR^{n \times m} &= O(n m) \text{ operations to compute} \\
		\text{Total: } O(n^2) + O(n^2) + O(n m) &= O(n^2 + n m) \text{ operations}
\end{align*}

We can see that if $m<n$ reverse-mode is more efficient, and if $m>n$ forward-mode is more efficient.

\end{document}
